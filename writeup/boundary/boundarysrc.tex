
\documentclass[a4paper,letterpaper,12pt,oneside,draft]{article}

\usepackage{geometry}
\geometry{margin=2cm,hoffset=0in, %
    headheight=0.5\baselineskip}
\usepackage{times}
\pagestyle{plain}
\usepackage{setspace}
\onehalfspacing
\usepackage{algorithm}
\usepackage{algpseudocode}
\usepackage{listings}

% setup file used by other files

\usepackage{amssymb} % Math package
\usepackage{amsmath} % Math package
\usepackage{amsthm}  % Math package
\usepackage{amsbsy} %For better bolding\usepackage{verbatim}
\usepackage{cancel}

\usepackage{chngcntr}
%\usepackage[numbers]{natbib}
%\usepackage{tocstyle}
\usepackage{xcolor}
\usepackage{regexpatch}
\usepackage[nocompress]{cite} %Correct ordering of citations in-text
%\usepackage{etoolbox}

\usepackage{csquotes}
\MakeOuterQuote{"}

\usepackage{mathrsfs}

\usepackage[obeyDraft, colorinlistoftodos]{todonotes}
\makeatletter
\xpatchcmd{\@todo}{\setkeys{todonotes}{#1}}{\setkeys{todonotes}{inline,#1}}{}{}
\makeatother

\usepackage{varioref}
\usepackage[pagebackref]{hyperref}
\usepackage[capitalize]{cleveref}

\crefformat{equation}{Eq.~#2\textup{#1}#3}
\crefmultiformat{equation}%
{Eqs.~#2\textup{#1}#3}% % first item in list
{ and~#2\textup{#1}#3}% % second (if exactly two items)
{, #2\textup{#1}#3}%     % middle (if more than two items)
{ and~#2\textup{#1}#3}  % last   (if more than two items)
\crefrangeformat{equation}%
{Eqs.~#3\textup{#1}#4 through~#5\textup{#2}#6}

\newcommand{\eec}{\;,}
\newcommand{\eep}{\;.}
\newcommand{\zth}{0th }
\newcommand{\fst}{1st }
\newcommand{\snd}{2nd }
\newcommand{\OpL}{\mathscr{L}}
\newcommand{\OpT}{\mathscr{T}}
\newcommand{\MG}{MG }

\newcommand{\allspace}{\ensuremath{\mathbb{R}^3}}
\newcommand{\norm}[1]{\left| #1 \right|}
\newcommand{\bracket}[1]{\ensuremath{\left\langle #1 \right\rangle}}
\newcommand{\bracketv}[1]{\bracket{#1}_V}
\newcommand{\bracketvo}[1]{\ensuremath{\bracket{#1}_{V,\Omega}}}
\newcommand{\bracketR}[1]{\ensuremath{\bracket{#1}_{\allspace}}}
\newcommand{\bracketex}[2][V]{\ensuremath{\bracket{#2}_{#1}}}
\newcommand{\bracketsa}[1]{\bracketex[\allspace,\Omega]{#1}}
\newcommand{\rdotJ}[1][\allspace]{\bracketex[#1]{\vec{r}\cdot\vec{J}}}
\newcommand{\rsqp}{\bracketR{r^2\phi}}
\newcommand{\intg}[2][g]{\ensuremath{\int_{E_{#1}}^{E_{#1-1}} #2 dE}}
\newcommand{\intcg}[2][g]{\ensuremath{\int_{E_{#1}}^{\infty} #2 dE}}
\newcommand{\vr}{\ensuremath{\vec{r}}}
\newcommand{\dvr}{\left(\vr-\vr_0\right)}
\newcommand{\psif}[1][]{\psi(\vr,\Omega#1,E#1)}
\newcommand{\psifz}[1][]{\psi(\vr,\Omega#1,E#1;\vr_0)}
\newcommand{\dvrdotJ}{\bracketR{\dvr\cdot\vec{J}}}
\newcommand{\dvrdotJg}{\bracketR{\dvr\cdot\vec{J_g}}}
\newcommand{\dvrsqp}{\bracketR{\dvr^2\phi}}
\newcommand{\dvrsqpg}{\bracketR{\dvr^2\phi_g}}
\newcommand{\regint}[1]{\ensuremath{\int_{V_0} #1 dV_0}}

\counterwithin{equation}{section}

\title{Defining two region diffusion coefficients for the boundary source problem: Exploratory notes}
\author{Eshed Magali \\ Edward W. Larsen}
\date{\today}


\widowpenalty10000
\clubpenalty10000

\begin{document}
\maketitle
\section{Premise}
    We consider a half-plane infinite slab (for $x>0$), with an input incoming source at $x=0$, of strength $\psi_b(\Omega,E)$. 
    The half-plane is made of a non-fissile homogeneous material with a given set of macroscopic cross sections, with anisotropic scattering. 
    This can be written explicitly as:
    \begin{align}
    \nonumber
        \forall x>0,\quad \Omega\cdot\nabla\psif &+ \Sigma_t(E)\psif \eec \\ 
    \label{eq:BTE}
        =& \frac{1}{4\pi}\int_0^\infty\int_{4\pi}\Sigma_{s}(E'\to E,\Omega\cdot\Omega')\psif[']d\Omega'dE' \eec\\
    \label{eq:BTE:BC}
        \forall\  \Omega\cdot\hat{x}>0\  \forall y\  \forall z, \quad &
        \psi(0,y,z,\Omega,E) = \psi_b(\Omega,E)\eec \\\nonumber
        \lim\limits_{x\to\infty}&\psi(\vr,\Omega,E) = 0 \eep
    \end{align}
    The solution to the problem posed in \cref{eq:BTE,eq:BTE:BC} is symmetric to translations in $y,z$, since the boundary conditions and cross sections are all symmetric to those same translations.
    
    Say we are interested in modeling this system using only two sets of multigroup diffusion parameters. Specifically, let $d$ such that the entire half plane $x>0$ is split into two regions, $x\leq d$ and $x>d$.
    We intend to use two sets of multigroup cross sections and diffusion coefficients, one for each of those regions. Once those parameters are set, this would define the following diffusion problem:
    \begin{align}
        \label{eq:Diff}
            \forall x>0, \quad-\nabla D_g(\vr)\nabla\phi_g(\vr) + \Sigma_{t,g}(\vr)\phi_g(\vr) &= \sum_{g'=1}^G\Sigma_{s0}^{g'\to g}(\vr)\phi_{g'}(\vr) \eec\\
        \label{eq:Diff:BC}
            \frac{1}{4}\phi_g(0) - \frac{1}{2}D(0)\nabla\phi_g(0) &= \intg{\int_{\Omega\cdot\hat{x}>0}\Omega\psi_b(\Omega,E)d\Omega} \eec\\
            \lim\limits_{x\to\infty}\phi_g(x) &= 0 \eec\\
        \label{eq:Diff:IC}
            \phi_g(d^-) &= \phi_g(d^+) \eec\\
            \nonumber
            D(d^-)\frac{d\phi_g}{dx}(d^-) &= D(d^+)\frac{d\phi_g}{dx}(d^+)\eec
    \end{align}
    where all of the multigroup parameters are histogram functions in space.
    For example:
    \begin{equation}
        \label{eq:TotalXS}
        \Sigma_{t,g}(\vr) = \begin{cases} \Sigma_{t,g}^1 & x\leq d \\ \Sigma_{t,g}^2 & x>d\end{cases}\eep
    \end{equation}
    
    For now, let us assume flux weighted homogenization for the cross sections. Specifically, we assume:
    \begin{align}
        \phi_{g}^1 &= \int_{0}^{d}\intg{\phi(x,E)}dx \eec\\
        \phi_{g}^2 &= \int_{d}^{\infty}\intg{\phi(x,E)}dx \eec\\
        \Sigma_{t,g}^1 &= \frac{1}{\phi_g^1}\int_{0}^{d}\intg{\Sigma_t(E)\phi(x,E)}dx \eec\\
        \Sigma_{t,g}^2 &= \frac{1}{\phi_g^2}\int_{d}^{\infty}\intg{\Sigma_t(E)\phi(x,E)}dx \eec\\
        \Sigma_{s0,1}^{g'\to g} &= \frac{1}{\phi_g^1} \int_{0}^{d}\intg{\intg[g']{\Sigma_{s0}(E'\to E)\phi(x,E)}'}dx \eec\\
        \Sigma_{s0,2}^{g'\to g} &= \frac{1}{\phi_g^2} \int_{d}^{\infty}\intg{\intg[g']{\Sigma_{s0}(E'\to E)\phi(x,E)}'}dx \eec
    \end{align}
    where $\phi(x,E)$ is the scalar flux solved from \cref{eq:BTE,eq:BTE:BC}.
    Our objective is to find 2 homogenized diffusion coefficients, $D_1,D_2$, such that the 2-region homogenized diffusion problem will preserve the physics of the transport problem.
    Our hope is that using something similar to CMM, where we preserve \rdotJ would yield good results. 
    
    The result of our derivation should reduce to the CMM results for $d\to\infty$:
    \begin{equation*}
        D^1_g = \frac{\intg{\rdotJ(E)}}{3\intg{\bracket{\phi}(E)}} \eec
    \end{equation*}
    and if $d\to0$ we expect a similar value for $D_g^2$.
    
    Our physical quantity of interest would be denoted $\Phi_{g^i}$. The transport value for this quantity would be defined as:
    \begin{align}
        \label{eq:Phi1:Transport}
        \Phi_{g,t}^1 &= \iint_{-\infty}^{\infty}\int_{0}^{d}\intg{\vr\cdot \vec{J}(x,E)}dxdydz \eec \\\nonumber
        \Phi_{g,t}^2 &= \iint_{-\infty}^{\infty}\int_{d}^{\infty}\intg{\vr\cdot \vec{J}(x,E)}dxdydz \eec
    \end{align}
    where $\vec{J}(x,E)$ is the neutron current for the solution of \cref{eq:BTE,eq:BTE:BC}.
    Notice that we don't really need to do this with the point-source breakdown because of symmetry to translations across the left boundary.
    For rigor this writeup is included.
    We start by looking at all point boundary source problems:
    \begin{align}
    \nonumber
    \forall x>0,\quad \Omega\cdot\nabla\psifz &+ \Sigma_t(E)\psifz \eec \\ 
    \label{eq:BTE:PS}
    =& \frac{1}{4\pi}\int_0^\infty\int_{4\pi}\Sigma_{s}(E'\to E,\Omega\cdot\Omega')\psifz[']d\Omega'dE' \eec\\
    \label{eq:BTE:PS:BC}
    \forall\  \Omega\cdot\hat{x}>0\  \forall y\  \forall z, \quad &
    \psi(0,y,z,\Omega,E;\vr_0) = \begin{cases}
    \psi_b(\Omega,E) & y=y_0, z=z_0 \\ 0 & \text{otherwise}
    \end{cases}\eec \\\nonumber
    \lim\limits_{\vr\to\infty}&\psi(\vr,\Omega,E) = 0 \eep
    \end{align}
    which allows us to define the point source current $J(\vr,E;\vr_0)$ as usual. We then define the fully rigorous quantity as:
    \begin{align}
    \label{eq:Phi1:Transport:Full}
        \Phi_{g,t}^1 &= \frac{1}{YZ} \int_{-\frac{Z}{2}}^{\frac{Z}{2}} \int_{-\frac{Y}{2}}^{\frac{Y}{2}} \iint_{-\infty}^{\infty} \int_{0}^{d} \intg{\dvr\cdot \vec{J}(\vr,E;\vr_0)}dxdydzdy_0dz_0 \eec \\\nonumber
        \Phi_{g,t}^2 &= \frac{1}{YZ} \int_{-\frac{Z}{2}}^{\frac{Z}{2}} \int_{-\frac{Y}{2}}^{\frac{Y}{2}} \iint_{-\infty}^{\infty} \int_{d}^{\infty} \intg{\dvr\cdot \vec{J}(\vr,E;\vr_0)}dxdydzdy_0dz_0 \eep
    \end{align}
    We will never need to simplify or find other forms for these quantities, and we can simply assume they are given from a MC simulation of the problem or otherwise obtained.
    
    For the diffusion problem we can similarly do a point-source breakdown. We define the point source diffusion equations as:
    \begin{align}
    \label{eq:Diff:PS}
    \forall x>0, \quad-\nabla D_g(\vr)\nabla\phi_g(\vr;\vr_0) + \Sigma_{t,g}(\vr)\phi_g(\vr;\vr_0) &= \sum_{g'=1}^G\Sigma_{s0}^{g'\to g}(\vr)\phi_{g'}(\vr;\vr_0) \eec\\
    \label{eq:Diff:PS:BC}
    \frac{1}{4}\phi_g(0,y,z;\vr_0) - \frac{1}{2}D^1\nabla\phi_g(0,y,z;\vr_0) &= \begin{cases}
    J^+ & y=y_0, z=z_0 \\
    0 & \text{otherwise}
    \end{cases} \eec\\\nonumber
    \lim\limits_{\vr\to\infty}\phi_g(\vr;\vr_0) &= 0 \eec\\
    \label{eq:Diff:PS:IC}
    \phi_g(d^-,y,z;\vr_0) &= \phi_g(d^+,y,z;\vr_0) \eec\\
    \nonumber
    D^1\frac{d\phi_g}{dx}(d^-,y,z;\vr_0) &= D^2\frac{d\phi_g}{dx}(d^+,y,z;\vr_0)\eec
    \end{align}
    The diffusion value for the quantity of interest similar to \cref{eq:Phi1:Transport:Full} would then be:
    \begin{align}
        \label{eq:Phi1:Diffusion}
        \Phi_{g,d}^1 &= \frac{1}{YZ} \int_{-\frac{Z}{2}}^{\frac{Z}{2}} \int_{-\frac{Y}{2}}^{\frac{Y}{2}} \iint_{-\infty}^{\infty} \int_{0}^{d} \dvr\cdot \vec{J}_g(\vr;\vr_0)dxdydzdy_0dz_0 \eec \\\nonumber
        \Phi_{g,d}^2 &= \frac{1}{YZ} \int_{-\frac{Z}{2}}^{\frac{Z}{2}} \int_{-\frac{Y}{2}}^{\frac{Y}{2}} \iint_{-\infty}^{\infty} \int_{d}^{\infty} \dvr\cdot \vec{J}_g(\vr;\vr_0)dxdydzdy_0dz_0 \eec
    \end{align}
    where $J_g(\vr;\vr_0)=-D_g(\vr)\nabla\phi(\vr;\vr_0)$ is the current of the diffusion problem in \cref{eq:Diff:PS,eq:Diff:PS:BC,eq:Diff:PS:IC}. 
    
\section{Semi-Analytic solution}
    Suppose the diffusion coefficients were given. We will now find an analytical expression for $\Phi_{g,d}^{i}$ for $i\in\{1,2\}$.
    Notice that for each of the two regions we can write using \cref{eq:Diff}:
    \begin{equation}
        \label{eq:Diff:diffeqsim}
        \frac{d^2\phi_g}{dx^2}(x) = D_g^{-1}\left[\Sigma_{t,g}\phi_g(x) - \sum_{g'=1}^G\Sigma_{s0}^{g'\to g}\phi_{g'}(x)\right]\eec
    \end{equation}
    which can be rewritten in matrix form as:
    \begin{equation}
    \label{eq:Diff:diffeqmat}
    \frac{d^2\vec{\phi}}{dx^2}(x) = D^{-1}\left[\Sigma_{t} - \Sigma_{s0}\right]\vec{\phi}(x)\eep
    \end{equation}
    where $D,\Sigma_t$ are diagonal matrices, and $\Sigma_{s0}$ is a full matrix. 
    These matrices are constant within each region.
    We introduce a simpler notation:
    \begin{equation}
    \label{def:U}
        U^i = \sqrt{(D^i)^{-1}\left(\Sigma_t^i - \Sigma_{s0}^i\right)}\eep
    \end{equation}
    
    \cref{eq:Diff:diffeqmat} is solvable. It has two linearly independent solutions\footnote{If $U^i$ is the unique root of $(D^i)^{-1}\left(\Sigma_t^i - \Sigma_{s0}^i\right)$. In these notes this is simply assumed.}, which can be written as:
    \begin{equation}
    \vec{\phi^i} = e^{-xU^i}\vec{A_i} + e^{xU^i}\vec{B_i} \eec
    \end{equation}
    where $\vec{A_i},\vec{B_i}$ will be uniquely defined by the boundary conditions and interface conditions.
    This is done by writing the four boundary conditions and interface conditions:
    \begin{align}
        \label{eq:BC:Left}
            \vec{A_1}+\vec{B_1} - 3D^1U^1\left(\vec{B_1}-\vec{A_1}\right) &= \intg{\int_{\Omega\cdot\hat{x}>0}\Omega\psi_b(\Omega,E)d\Omega} \eec \\
        \label{eq:BC:Right}
            \vec{B_2} &= 0 \eec \\
        \label{eq:IC:Flux}
            e^{-dU^1}\vec{A_1} + e^{dU^1}\vec{B_1} &= e^{-dU^2}\vec{A_2} + e^{dU^2}\vec{B_2} \eec \\
        \label{eq:IC:Current}
            D^1U^1\left(e^{dU^1}\vec{B_1} - e^{-dU^1}\vec{A_1}\right) &= D^2U^2\left(e^{dU^2}\vec{B_2} - e^{-dU^2}\vec{A_2}\right) \eep
    \end{align}
    
    With $\vec{A_i},\vec{B_i}$ uniquely defined by the solution these linear equations, we can obtain $\Phi_{g,d}^i$. For example, in the first region\footnote{This should be a 3D integral, probably, and the surface areas would cancel out. Todo in next version}:
    \begin{align}
    \label{eq:Res:1}
        \vec{\Phi}_{d}^1 &= -\int_{0}^{d}x\cdot D^1\nabla \phi(x)dx \\\nonumber
        &= -D^1\int_{0}^{d}\left[\nabla\cdot (x\vec{\phi}) - 3\vec{\phi}\right]dx \\\nonumber
        &= 3D^1\int_{0}^{d}\vec{\phi}dx - D^1d\vec{\phi}(d) \eep
    \end{align}
    
    This can be immediately solved for. $\int_0^d\phi(x)dx$ is known from the transport solution of \cref{eq:BTE,eq:BTE:BC}, and the other term is the evaluated term:
    \begin{equation}
    \label{eq:interface:intro}
        \vec{\phi}(d) = e^{-dU^2}\vec{A_2}\eep
    \end{equation}
    
    For the second region we similarly get:
    \begin{align}
     \label{eq:Res:2}
         \vec{\Phi}_{d}^2 &= -\int_{d}^{\infty}x\cdot D^2\nabla \phi(x)dx \\\nonumber
         &= -D^2\int_{d}^{\infty}\left[\nabla\cdot (x\vec{\phi}) - 3\vec{\phi}\right]dx \\\nonumber
         &= 3D^2\int_{d}^{\infty}\vec{\phi}dx + D^2d\vec{\phi}(d) \eep
    \end{align}
    
    We can now iteratively search for the diffusion coefficient matrices $D^1,D^2$ for which: 
    \begin{align*}
    \vec{\Phi}_d^1 &= \intg{\int_0^d x\cdot J(x,E)dx} \eec \\
    \vec{\Phi}_d^2 &= \intg{\int_d^\infty x\cdot J(x,E)dx} \eec
    \end{align*}
    where again $J(x,E)$ is given from the solution to \cref{eq:BTE,eq:BTE:BC}.
    It isn't obvious that such diagonal matrices exist. It is also not obvious that any specific non-linear solver would converge to the correct solution even if it exists. What is clear, however, is that given both $d\to\infty$ or $d\to 0$, the result agrees with the known CMM result for which:
    
    \begin{equation}
    D_g = \frac{\Phi_{g,t}}{3\int\intg{\phi(\vr)}dV}\eep
    \end{equation}
    
    Another important conclusion is that preservation of $\vec{\Phi}$ will result in a higher diffusion coefficient in $0\leq x\leq d$, and a lower diffusion coefficient in $x>d$, as is obvious from \cref{eq:Res:1,eq:Res:2}.
    
\section{Algorithm}
    The preceding section describes a way to define diffusion based migration moments once the diffusion coefficients are known.
    Thus, we can perform a residual minimization algorithm to try and find a set of diagonal diffusion coefficient matrices that would minimize the residual (meaning they would preserve the required moment).
    The algorithm goes as follows:
    \begin{algorithm}\caption{General algorithm}\label{alg:gen}
    \begin{algorithmic}[1]
        \Require $\Sigma_{t}^1,\Sigma_{t}^2,\Sigma_{s0}^1,\Sigma_{s0}^2,D^0,J^+_{bc},\epsilon, \Phi_{t}^1,\Phi_{t}^2, \phi^1,\phi^2$
        \State $D^1,D^2 \gets D^0$
        \State $r \gets \text{ones}(G,2)$
        \While{any($r>\epsilon$)}
            \State $U^1 \gets \sqrt{\left(D^1\right)^{-1}\left(\Sigma_{t}^1-\Sigma_{s0}^1\right)}$
            \State $U^2 \gets \sqrt{\left(D^2\right)^{-1}\left(\Sigma_{t}^2-\Sigma_{s0}^2\right)}$
            \State Solve \cref{eq:BC:Left,eq:BC:Right,eq:IC:Flux,eq:IC:Current} for $A_2$
            \State $\Phi_{d}^1 \gets 3D^1\phi^1 - D^1d\exp(-dU^2)A_2$
            \State $\Phi_{d}^2 \gets 3D^2\phi^2 + D^2d\exp(-dU^2)A_2$
            \State $r(:,1) \gets \Phi_{d}^1 - \Phi_{t}^1$
            \State $r(:,2) \gets \Phi_{d}^2 - \Phi_{t}^2$
            \State Make new guess for $D^1,D^2$
        \EndWhile
    \end{algorithmic}
    \end{algorithm}
    
    One could immediately see that if we define statements 4 through 10 as a function that returns $r$ as based on the value of $D^2,D^2$, with the rest of the data used as constant parameters. The algorithm as a whole then becomes a root search problem for a non-linear function. This can be summarized as:
    
    \begin{algorithm}[H]\caption{Algorithm as a root search}\label{alg:root}
    \begin{algorithmic}[1]
        \Require $\Sigma_{t}^1,\Sigma_{t}^2,\Sigma_{s0}^1,\Sigma_{s0}^2,D^0,J^+_{bc},\epsilon, \Phi_{t}^1,\Phi_{t}^2, \phi^1,\phi^2$
        \Function{DiffResidual}{$D$}
        \State $D^1 \gets D(:,1)$
        \State $D^2 \gets D(:,2)$
        \State $U^1 \gets \sqrt{\left(D^1\right)^{-1}\left(\Sigma_{t}^1-\Sigma_{s0}^1\right)}$
        \State $U^2 \gets \sqrt{\left(D^2\right)^{-1}\left(\Sigma_{t}^2-\Sigma_{s0}^2\right)}$
        \State Solve \cref{eq:BC:Left,eq:BC:Right,eq:IC:Flux,eq:IC:Current} for $A_2$
        \State $\Phi_{d}^1 \gets 3D^1\phi^1 - D^1d\exp(-dU^2)A_2$
        \State $\Phi_{d}^2 \gets 3D^2\phi^2 + D^2d\exp(-dU^2)A_2$
        \State $r(:,1) \gets \Phi_{d}^1 - \Phi_{t}^1$
        \State $r(:,2) \gets \Phi_{d}^2 - \Phi_{t}^2$
        \State \Return r
        \EndFunction
        
        \State \Call {FindRoot}{DiffResidual, $x_0=D^0$, convergence$=\epsilon$}
    \end{algorithmic}
    \end{algorithm}
    
\appendix
\section{Appendix: Python Implementation}
    This algorithm is of course fairly easy to implement as well. A Python implementation is attached below. Currently this is a draft so the code isn't included to save the reader the time.
    
    \lstinputlisting[language=Python, breaklines=true]{../../src/util.py}
    \lstinputlisting[language=Python, breaklines=true]{../../src/boundary.py}
    
\end{document}